\thispagestyle{empty}
\begin{center}
    意味情報を考慮した熱赤外線画像着色に関する研究
\end{center}

\thispagestyle{empty}
熱赤外線(Thermal Infrared, TIR)カメラは温度に応じて物体の表面から放射される熱赤外線を捉えるカメラであり,雨や霧,砂嵐などの悪天候下,夜間やトンネル内などの低照度環境下でも明瞭な画像が撮影できることから,様々な領域で応用されている.例えば自動運転の分野においてはタクシー用のセンサとして採用された例があるほか,航空分野では熱赤外線映像を視界に重畳して操縦時の状況認識を補助するシステムが利用されている.また,多様な状況に適応する必要がある海難救助の現場において要救助者の捜索用装備としても活用されている.
その一方で,熱赤外線カメラで撮影された画像は可視光画像のような豊富な色情報を持たないという欠点があるため、状況の解釈が可視光画像の解釈よりも難しくなっている。これは、熱赤外線カメラの広範な普及を阻害する要因の一つになっているほか、熱赤外線画像を可視光画像用のコンピュータビジョンアルゴリズムに転用することを妨げている。


そこで、近年ではニューラルネットワークを用いて熱赤外線画像を着色し,擬似的な可視光カラー画像を生成することで上記の問題を解決しようとするアプローチが取られている.これにより、熱赤外線画像を可視光カラー画像として扱うことが可能になり、ユーザーは熱赤外線画像特有の"状況変化に対する頑健さ"と可視光カラー画像特有の"豊富な色情報による理解の容易さ"の両方の恩恵を受けることができる。しかし,既存手法で生成された画像は前景物体が背景領域と同化してしまうなど不自然な見た目になってしまうことも多く,依然として画像の品質に課題がある.

本論文ではリアルな画像の生成が可能な敵対的生成ネットワークにセマンティックセグメンテーション用のモデルを組み合わせることで,従来よりも熱赤外線画像を高品質に着色可能な画像生成手法を提案する.セマンティックセグメンテーションとは画像を物体の意味に応じて領域分割するタスクである.このタスクのモデルを用いて入力画像中の物体の意味情報に注目した特徴量を抽出し,補助的な入力として利用することで入力画像に対する着色モデルの意味理解を補強する.また,生成されるラベルマップを用いて「車」「人物」等の特定のクラスに属する領域を切り抜き,該当する領域のみについて敵対的損失を計算することで生成画像の局所的な品質を高める.

提案手法は2つのモジュールから構成される生成器と、2種類の判別器から構成される敵対的生成ネットワークである。
生成器はカラー画像を生成する着色モジュールと、入力画像に対してセマンティックセグメンテーションを行うセグメンテーションモジュールから成る。
着色モジュールはU-Net構造のネットワークで、グレースケールの熱赤外線画像とセグメンテーションモジュールで抽出された特徴量を受け取り、RGB形式の着色画像を生成する。
セグメンテーションモジュールは学習済み・重み固定のセマンティックセグメンテーションモデルであり、熱赤外線画像を受け取って同解像度のセマンティックラベルマップを出力する。また、その過程で入力画像から抽出した特徴量と生成したラベルを着色モジュールに伝達し、着色モジュールの入力画像に対する意味理解を補強することで、生成画像の品質を高める。
生成画像の真贋判定には画像全体の真贋判定を行うグローバル画像判別器に加え、セマンティックラベルマップに基づき一つのクラスだけを抽出した画像に対して適用するクラス画像判別器を使用し、生成画像の局所的な品質を高める。
損失関数には生成画像とGround-TruthのL1ノルムから計算されるコンテンツ損失、グローバル画像判別器から計算される敵対的損失、高レベル特徴を再現するための知覚的損失、ノイズ抑制のためのTV損失に加え、クラス画像判別器から計算されるクラス敵対的損失を追加する。2種類の敵対的損失には生成画像とGround-Truthの相対的な本物らしさを元に損失を計算するRSGANを使用し、学習安定化と生成画像の品質向上を図っている。

複数の熱赤外線画像データセットを用いた実験を通じて、提案手法が既存手法と同等、またはそれを上回る品質の画像を生成できることを示した。一部データセットのPSNRとSSIMは提案手法の方が既存手法よりも低くなったが、知覚的品質と相関が高いLPIPSとFIDでは全てのデータセットで提案手法が最も優れた結果を示した。また、既存手法で意味的な混乱が生じている物体領域を、提案手法では自然な色とテクスチャで再現できていることを定性的に確認した。
%セグメンテーションモジュールとクラス画像判別器に対するアブレーションスタディを行い、提案する2つのコンポーネントが着色の品質向上に有効であることを示した。更に、様々な敵対的損失の計算方法を提案手法に適用し、着色に最適な損失計算の方法について考察を行った。
提案手法は優れた生成結果を示した一方で、一部のテスト画像に対して適切に着色を行うことができず、データの選択や入力方法について更なる改良が必要であることが示された。今後の課題は、品質を保ちつつ計算コストを低くできるネットワークの構造や、着色に最適な入力画像に対する処理方法を検討することである。


%\noindent 熱赤外線(Thermal Infrared, TIR)カメラは温度に応じて物体の表面から放射される熱赤外線を捉えるカメラであり,雨や霧,砂嵐などの悪天候下,夜間やトンネル内などの低照度環境下でも明瞭な画像が撮影できることから,様々な領域で応用されている.例えば自動運転の分野においてはタクシー用のセンサとして採用された例があるほか,航空分野では熱赤外線映像を視界に重畳して操縦時の状況認識を補助するシステムが利用されている.また,多様な状況に適応する必要がある海難救助の現場において要救助者の捜索用装備としても活用されている.\par
%その一方で,熱赤外線カメラで撮影された画像は可視光画像のような豊富な色情報を持たないという欠点があるため、状況の解釈が可視光画像の解釈よりも難しくなっている。これは、熱赤外線カメラの広範な普及を阻害する要因の一つになっているほか、熱赤外線画像を可視光画像用のコンピュータビジョンアルゴリズムに転用することを妨げている。


%そこで、近年ではニューラルネットワークを用いて熱赤外線画像を着色し,擬似的な可視光カラー画像を生成することで上記の問題を解決しようとするアプローチが取られている.これにより、熱赤外線画像を可視光カラー画像として扱うことが可能になり、ユーザーは熱赤外線画像特有の"状況変化に対する頑健さ"と可視光カラー画像特有の"豊富な色情報による理解の簡単さ"の両方の恩恵を受けることができる。

%しかし,既存手法で生成された画像は前景物体が背景領域と同化してしまうなど不自然な見た目になってしまうことも多く,依然として画像の品質に課題がある.\par

%本論文ではリアルな画像の生成が可能な敵対的生成ネットワークにセマンティックセグメンテーション用のモデルを組み合わせることで,従来よりも熱赤外線画像を高品質に着色可能な画像生成手法を提案する.セマンティックセグメンテーションとは画像を物体の意味に応じて領域分割するタスクである.このタスクのモデルを用いて入力画像中の物体の意味情報に注目した特徴量を抽出し,補助的な入力として利用することで入力画像に対する着色モデルの意味理解を補強する.また,生成されるラベルマップを用いて「車」「人物」等の特定のクラスに属する領域を切り抜き,該当する領域のみについて敵対的損失を計算することで生成画像の局所的な品質を高める.
